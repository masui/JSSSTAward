\documentclass{compsoft}

% Preamble
%
% 「コンピュータソフトウェア」誌に掲載される論文の場合,次で
% 巻数,号数,開始ページ,終了ページを指定する.
\volNoPp{29}{1}{78}{84}

% ここに,使用するパッケージを列挙する.
\usepackage[dvips]{graphics}

% ユーザが定義したマクロなどはここに置く.ただし学会誌のスタイルの
% 再定義は原則として避けること.

\begin{document}

% 論文のタイトル
\title{ユニバーサルなユーザインタフェース}

% 著者
% 和文論文の場合,姓と名の間には半角スペースを入れ,
% 複数の著者の間は全角スペースで区切る
%
\author{増井 俊之
%
% ここにタイトル英訳 (英文の場合は和訳) を書く.
%
\ejtitle{Concurrent Operations on Splay Trees.}
%
% ここに著者英文表記 (英文の場合は和文表記) および
% 所属 (和文および英文) を書く.
% 複数著者の所属はまとめてよい.
%
\shozoku{Kazunori Ueda}{慶應義塾大学 環境情報学部}%
{Dept.\ of Information and Computer Science, Waseda University}
%
% 出典情報は \shutten とすれば出力される.
\shutten
%
% 受付年月日,記事カテゴリなどは自動的に生成される.
\uketsuke{2017}{1}{10}
%
% その他,脚注に入れるものがあれば,\note に記述する.
%\note{脚注に入れる内容}
}

%
% 和文アブストラクト
\Jabstract{%
ユーザインタフェースとは
}
%
% 英文アブストラクト(本サンプルの原論文にはなし)
\Eabstract{%
We talk ablut uniersal UI.
%
}
\maketitle

\section{はじめに}

\chosha{増井俊之}{
  1984年東京大学大学院工学系研究科電子工学専門課程修士課程修了。 工学博士。
  シャープ、ソニーコンピュータサイエンス研究所、産業技術総合研究所、Apple Inc.などに勤務後、
  2009年4月より慶應義塾大学]環境情報学部教授。
情報検索、テキスト入力、情報視覚化、実世界指向インタフェース、予測インタフェース、認証技術など、
ユーザインタフェースに関連する幅広い研究開発を行なっている。
携帯電話やスマートフォンで広く利用されている予測入力システムPOBoxや
フリック入力システムの開発者。
Gyazo, Scrapbox, Helpfeel, EpisoPass, 本棚.orgなど
各種のWebサービスを運用中。
WISSの話も書く
}

  
  昔のコンピュータは専門家が使うものでしたが、
  最近は普通の人々がパソコンやスマホを活用しています。

  誰もがコンピュータを使えるようになったのは
  計算機のユーザインタフェースの改良の結果です。
  昔のコンピュータはコマンドラインで操作するのが普通だった時代もありましたが、
  40年ほど前に
  ウィンドウやメニューなどを活用するグラフィカルユーザインタフェースが発明されてから
  誰もがコンピュータを使えるようになってきたといえるでしょう。



  
  
\end{document}
