%
% 講演資料
%  https://scrapbox.io/masui/ソフトウェア科学会_基礎研究賞講演_2021/9/3
% Cloud LaTeX
%  https://cloudlatex.io/projects/411855/edit
% 記事依頼
%  https://mail.google.com/mail/u/0/?zx=a0kk8p4fqhm3#inbox/WhctKKWxdNgXpJNJdpfrtHgVMQJTthSKpCLwgCTppBnkqJbWZmmsxwzHgMwmjcxmLrKmfZV
% 執筆要項 (スタイルファイル)
%  https://www.jssst.or.jp/edit/detail/style_files.html
% 記事例
%  https://s3-ap-northeast-1.amazonaws.com/masui.org/f/5/f5361cf93c65dd661e5797ed105bfeca.pdf
%

\documentclass[topics]{compsoft} % トピックス

% 「コンピュータソフトウェア」誌に掲載される論文の場合,次で巻数,号数,開始ページ,終了ページを指定する.
\volNoPp{29}{1}{78}{84}

\usepackage[dvips]{graphics}

\begin{document}

\title{ユニバーサルなユーザインタフェース}

% 著者
% 和文論文の場合,姓と名の間には半角スペースを入れ,
% 複数の著者の間は全角スペースで区切る
%
\author{増井 俊之
%
% ここにタイトル英訳 (英文の場合は和訳) を書く.
%
\ejtitle{Concurrent Operations on Splay Trees.}
%
% ここに著者英文表記 (英文の場合は和文表記) および
% 所属 (和文および英文) を書く.
% 複数著者の所属はまとめてよい.
%
\shozoku{Kazunori Ueda}{慶應義塾大学 環境情報学部}%
{Dept.\ of Information and Computer Science, Waseda University}
%
% 出典情報は \shutten とすれば出力される.
\shutten
%
% 受付年月日,記事カテゴリなどは自動的に生成される.
\uketsuke{2017}{1}{10}
%
% その他,脚注に入れるものがあれば,\note に記述する.
%\note{脚注に入れる内容}
}

\long\def\comment#1{}

\maketitle

\section{はじめに}
講演者は30年以上にわたり、コンピュータの使い勝手(ユーザインタフェース)を改善する
様々な研究開発を行なってきた。従来のコンピュータは主に専門家が使うものであったが、
現在は誰でもいつでもどこでもコンピュータやネットワークを使うようになっており、
弱者をサポートするためにコンピュータが使われる機会も増えてきている。
この分野の様々な研究開発の歴史及び将来の展望について述べる。

この度は,2020年度日本ソフトウェア科学会基礎研究賞を頂き,大変光栄に思います。
数多くの優秀な共同研究者に恵まれて,この度の受賞に至りました。
深く感謝申し上げます。
受賞の対象となった研究は、「実証的(エンピリカル)ソフトウェア工学」と「ソフトウェアプロテクション」という
2つの分野にまた
がる一連のテーマです。以降,受賞記念講演のスライドに沿って説明いたします。

\chosha{増井俊之}
{1984年東京大学大学院工学系研究科電子工学専門課程修士課程修了。
工学博士。
シャープ、ソニーコンピュータサイエンス研究所、産業技術総合研究所、Apple Inc.などに勤務後、
2009年4月より慶應義塾大学環境情報学部教授。
情報検索、テキスト入力、情報視覚化、実世界指向インタフェース、予測インタフェース、認証技術など、
ユーザインタフェースに関連する幅広い研究開発を行なっている。
携帯電話やスマートフォンで広く利用されている予測入力システムPOBoxや
フリック入力システムの開発者。
Gyazo, Scrapbox, Helpfeel, EpisoPass, 本棚.orgなど
各種のWebサービスを運用中。
WISSの話も書く
}

\end{document}
