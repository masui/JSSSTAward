%
% https://scrapbox.io/masui/ソフトウェア科学会_基礎研究賞講演_2021/9/3
%
% 
% 増井先生
% 
% 「コンピュータソフトウェア」の編集委員長をつとめております慶応大学の河野と申します.
% (大学は同じでも SFC と矢上だとなかなかお会いする機会がないですね).
% 
% この度は基礎研究賞受賞おめでとうございます.特別講演,楽しく拝聴させていただきました.
% 「コンピュータソフトウェア」では,毎年,基礎研究賞の受賞者に特別講演の内容をまとめたトピックス記事の執筆をお願いしております.今回も,是非,お願いしたいと思っております.
% 
% 文章を 6 ページ書いていただくのは大変かと思いますので,スライドの内容を貼り付けて,
% それにノートをつけるような形でも問題ありません.これまでにもそういう形で書いていただいております.
% 参考のため,昨年のものを 1 つ添付しておきます.
% 
% トピックス記事は 6 ページとなります.原稿の締め切りは 11 月初旬となります.
% 執筆要項等は以下の URL から参照できます.
% https://www.jssst.or.jp/edit/detail/style_files.html
% 
% お忙しいところ申し訳ございませんが,是非ともお引き受けいただきたく,お願い申し上げます.

\documentclass{compsoft}

% Preamble
%
% 「コンピュータソフトウェア」誌に掲載される論文の場合,次で
% 巻数,号数,開始ページ,終了ページを指定する.
\volNoPp{29}{1}{78}{84}

% ここに,使用するパッケージを列挙する.
\usepackage[dvips]{graphics}

% ユーザが定義したマクロなどはここに置く.ただし学会誌のスタイルの
% 再定義は原則として避けること.

\begin{document}

% 論文のタイトル
\title{ユニバーサルなユーザインタフェース}

% 著者
% 和文論文の場合,姓と名の間には半角スペースを入れ,
% 複数の著者の間は全角スペースで区切る
%
\author{増井 俊之
%
% ここにタイトル英訳 (英文の場合は和訳) を書く.
%
\ejtitle{Concurrent Operations on Splay Trees.}
%
% ここに著者英文表記 (英文の場合は和文表記) および
% 所属 (和文および英文) を書く.
% 複数著者の所属はまとめてよい.
%
\shozoku{Kazunori Ueda}{慶應義塾大学 環境情報学部}%
{Dept.\ of Information and Computer Science, Waseda University}
%
% 出典情報は \shutten とすれば出力される.
\shutten
%
% 受付年月日,記事カテゴリなどは自動的に生成される.
\uketsuke{2017}{1}{10}
%
% その他,脚注に入れるものがあれば,\note に記述する.
%\note{脚注に入れる内容}
}

%
% 和文アブストラクト
\Jabstract{%
講演者は30年以上にわたり、コンピュータの使い勝手(ユーザインタフェース)を改善する
様々な研究開発を行なってきた。従来のコンピュータは主に専門家が使うものであったが、
現在は誰でもいつでもどこでもコンピュータやネットワークを使うようになっており、
弱者をサポートするためにコンピュータが使われる機会も増えてきている。
この分野の様々な研究開発の歴史及び将来の展望について述べる。
}
%
% 英文アブストラクト(本サンプルの原論文にはなし)
\Eabstract{%
We talk ablut uniersal UI.}
%
\maketitle

\section{はじめに}

\chosha{増井俊之}
{1984年東京大学大学院工学系研究科電子工学専門課程修士課程修了。
工学博士。
シャープ、ソニーコンピュータサイエンス研究所、産業技術総合研究所、Apple Inc.などに勤務後、2009年4月より慶應義塾大学]環境情報学部教授。情報検索、テキスト入力、情報視覚化、実世界指向インタフェース、予測インタフェース、認証技術など、ユーザインタフェースに関連する幅広い研究開発を行なっている。
携帯電話やスマートフォンで広く利用されている予測入力システムPOBoxや
フリック入力システムの開発者。
Gyazo, Scrapbox, Helpfeel, EpisoPass, 本棚.orgなど
各種のWebサービスを運用中。
WISSの話も書く
}

\end{document}
